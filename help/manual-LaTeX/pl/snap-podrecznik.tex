% !TeX spellcheck = pl

\documentclass[a4paper]{report}

\input{../common/defs.tex}

\usepackage[polish]{babel}
\usepackage{polski}
\frenchspacing
\usepackage{indentfirst}

\begin{document}

\title{Snap! --- Podręcznik użytkownika}
\author{Brian Harvey\texorpdfstring{ \and}{,} Jens M\"onig}
\date{}

\manualtitlepage[Tłumaczenie na język polski:\\Bartosz Leper]{Snap!\\Podręcznik użytkownika}

\tableofcontents

\chapter*{}
\section*{Podziękowania}

Mieliśmy ogromne szczęście do mentorów. Jens zdobył dużo doświadczenia pracując wśród pionierów Smalltalka: Alana Kaya, Dana Ingallsa i~reszty ekipy, która wynalazła komputery osobiste i~programowanie obiektowe w~najlepszych dniach firmy Xerox PARC. Pracował z~Johnem Maloneyem z~zespołu Scratcha w~MIT\footnote{Massachusetts Institute of Technology, amerykańska uczelnia techniczna --- przyp. tłum.}, autorem platformy graficznej Morphic, wciąż stanowiącej fundament \Snap{a}. Znakomity projekt języka Scratch, autorstwa Lifelong Kindergarten Group z~MIT Media Lab, odgrywa w~\Snap{ie} kluczową rolę.

\textbf{\emph{Nasza poprzednia wersja, BYOB, była bezpośrednią modyfikacją kodu źródłowego Scratcha. \Snap{} został napisany od zera, lecz struktura jego kodu oraz interfejs użytkownika pozostają mocno zakorzenione w~Scratchu. Z~kolei zespół Scratcha, który mógłby widzieć w~nas rywali, przyjął nas ciepło i~okazał nam całkowite wsparcie.}}

Brian zdobywał szlify w~MIT oraz Stanford Artificial Intelligence Labs\footnote{Laboratorium sztucznej inteligencji na Uniwersytecie Stanforda --- przyp. tłum.}, gdzie uczył się pod okiem Johna McCarthy'ego, twórcy Lispa, oraz Geralda~J. Suss\-mana i~Guya Steele'a, twórców języka Scheme. Zdobywał również wiedzę od wielu innych wybitnych informatyków, w~tym autorów najlepszej książki z zakresu informatyki --- \emph{Struktury i~interpretacji programów komputerowych}: Hala Abelsona, Geralda~J. Suss\-mana i~Julie Suss\-man.

\textbf{\emph{Za starych dobrych czasów mawialiśmy w~MIT Logo Lab: ,,Język Logo to Lisp braniu BASIC-a''. Dziś, ze swoimi pierwszoklasowymi procedurami, zasięgami leksykalnymi~i pierwszoklasowymi kontynuacjami, \Snap{} jest jak Scheme w~przebraniu Scratcha.}}

Szczęśliwym zrządzeniem losu, poprzez forum Scratch Advanced Topics, poznaliśmy wspaniałą grupę błyskotliwych uczniów gimnazjów~(!\@) i liceów. Kilku z nich wniosło swój wkład w~kod \Snap{a}: Kartik Chandra, Nathan Dinsmore, Connor Hudson i~Ian Reynolds. Ponadto wielu zgłosiło pomysły i~raporty błędów podczas testowania wersji alfa. Wśród studentów Uniwersytetu Kalifornijskiego w~Berkeley, którzy przyczynili się do rozwoju kodu, znajdują się Michael Ball, Achal Dave, Kyle Hotchkiss, Ivan Motyashov i~Yuan Yuan. Wymienianie wszystkich tłumaczy zajęłoby zbyt wiele miejsca, ale można ich odnaleźć w~okienku ,,O \Snap{}\ldots'' dostępnym w~programie.

Niniejsze dzieło powstało częściowo w~ramach grantu nr~1143566 udzielonego przez National Science Foundation, a częściowo przy wsparciu firmy MioSoft.

\clearpage

\begin{center}
\bf \Huge \Snap{} \\
Podręcznik użytkownika \\
\huge Wersja 4.0 \vspace{40pt}
\end{center}

\Snap{} to rozszerzona reimplementacja języka Scratch (\url{http://scratch.mit.edu}), która pozwala na tworzenie własnych bloków (ang.\ \textit{Build Your Own Blocks}; stąd dawna nazwa \Snap{a} --- BYOB). Opisywany tu język obsługuje pierwszoklasowe listy, procedury i~kontynuacje. Te dodatkowe możliwości sprawiają, że nadaje się on do przeprowadzenia poważnego wstępu do informatyki dla uczniów liceów i szkół wyższych.

Aby uruchomić środowisko \Snap{}, wystarczy otworzyć przeglądarkę internetową i~wpisać adres \url{http://snap.berkeley.edu/run}, aby zacząć pracę z~minimalnym zestawem bloków. Można też użyć adresu \url{http://snap.berkeley.edu/init}, aby załadować niewielki zestaw dodatkowych bloków. Wiąże się to z~nieco wolniejszym ładowaniem, ale jest zalecane dla wygody użytkowników (w~dalszej części podręcznika będziemy zakładali korzystanie z~tej właśnie metody).

\clearpage

\chapter{Bloki, skrypty i~duszki}

W~tym rozdziale poznamy kilka cech języka \Snap{} odziedziczonych po Scratchu; doświadczeni użytkownicy Scratcha mogą przejść od razu do sekcji~\ref{sec:zagnieżdżanie-duszków}.

\Snap{} jest językiem programowania --- notacją, przy pomocy której możemy powiedzieć komputerowi, co ma zrobić. Jednak w~odróżnieniu od większości innych, \Snap{} jest językiem wizualnym; programując w~nim, zamiast posługiwać się klawiaturą, używamy metody ,,przeciągnij i~upuść'', dobrze znanej użytkownikom komputerów.

Uruchom teraz środowisko \Snap{}. Powinieneś zobaczyć ekran podzielony na kilka obszarów:\footnote{\onehalfspacing Pierwsze uruchomienie \Snap{a} prawdopodobnie spowoduje wyświetlenie angielskiej wersji programu; aby przełączyć się na język polski, należy kliknąć menu ,,Ustawienia''~\inlinepic{../common/btn-settings} na pasku narzędzi, a~następnie użyć polecenia ,,Language\ldots'' (,,Język\ldots'') --- przyp. tłum.}\nopagebreak

\begin{center}
\def\svgwidth{\textwidth}
\input{obszary-okna.pdf_tex}
\end{center}

(Proporcje tych stref mogą się różnić, w~zależności od rozmiaru i~kształtu okna przeglądarki).

Program w~języku \Snap{} składa się z~jednego lub więcej \emph{skryptów}, te zaś z~kolei --- z~\emph{bloków}. Oto przykładowy skrypt:\nopagebreak

\label{fig:typowy-skrypt}
\bigpic{typowy-skrypt}

Na powyższy skrypt składa się pięć bloków w~trzech różnych kolorach, odpowiadających trzem z~ośmiu \emph{palet} z~blokami. Obszar palet, znajdujący się po lewej stronie okna, pokazuje jedną paletę na raz. Do zmiany widocznej palety służy osiem przycisków znajdujących się tuż nad tym obszarem. Bloki ciemnożółte, widoczne w~naszym skrypcie, pochodzą z~palety ,,Kontrola''; zielone z~palety ,,Pisak'', a~niebieskie --- z~palety ,,Ruch''. Aby złożyć taki skrypt, należy poprzeciągać odpowiednie bloki z~palet do \emph{obszaru skryptów}, umiejscowionego na środku okna. Kiedy układamy jeden blok pod drugim w~taki sposób, aby wcięcie dolnego bloku znalazło się w~pobliżu wypustki tego powyżej, bloki łączą się ze sobą (ang. \textit{snap together}; stąd nazwa języka \Snap{}):\nopagebreak

\bigpic{laczenie-blokow}

Pozioma biała linia sygnalizuje, że jeśli puścimy zielony blok, połączy się on z~wypustką ciemnożółtego.

\subsection{Bloki-czapki i~bloki komend}

Na górze skryptu znajduje się \emph{blok-czapka}, który określa, kiedy skrypt ma zostać wykonany. Nazwy bloków-czapek zazwyczaj zaczynają się słowem ,,\code{kiedy}''; nasz przykładowy skrypt powinien zostać uruchomiony w~momencie kliknięcia zielonej flagi, znajdującej się w pobliżu prawej strony paska narzędzi \Snap{a}. (Pasek ten jest częścią okna programu \Snap{}; nie chodzi tutaj o pasek menu przeglądarki lub systemu operacyjnego). Skrypt nie musi posiadać czapki, jednak w~takim przypadku zostanie wykonany tylko wtedy, gdy użytkownik sam go kliknie. Skrypt nie może mieć więcej niż jednej czapki; jej charakterystyczny kształt służy łatwiejszemu zapamiętaniu tej szczególnej własności.

Pozostałe bloki w naszym skrypcie to \emph{bloki komend}. Każdy z~nich oznacza jakąś akcję, którą \Snap{} potrafi wykonać. Na przykład blok \inlinepic{przesun-o-10-krokow} nakazuje duszkowi\footnote{W grafice komputerowej słowem ,,duszek'' (ang. \textit{sprite}) nazywa się ruchomy obiekt na ekranie --- przyp. tłum.}, czyli strzałce na \emph{scenie} po prawej stronie okna, aby przesunął się o~dziesięć kroków do przodu w~kierunku, w~którym jest zwrócony. Każdy krok to niewielka odległość na ekranie. Wkrótce przekonamy się, że na scenie może być więcej duszków, a~każdy z nich może mieć własne skrypty. Ponadto duszki nie muszą wyglądać jak strzałki; ich kostiumy mogą być dowolnymi obrazkami. Kształt bloku \code{przesuń} ma za zadanie przypominać klocek, skrypt zaś jest jak wieża z klocków. Słowa ,,blok'' będziemy używać dla oznaczenia zarówno graficznego symbolu na ekranie, jak i~procedury (akcji) jaką ten blok wykonuje.

Liczbę $10$ w powyższym bloku \code{przesuń} nazywamy jego \emph{parametrem}. Kliknąwszy na białym, owalnym polu, możemy wpisać w~jej miejsce dowolną inną. W przykładowym skrypcie ze strony \pageref{fig:typowy-skrypt} parametrem jest liczba $100$. Jak się później okaże, pola parametrów mogą mieć kształty inne od owalnych; oznacza to wtedy, że akceptują one wartości inne niż liczby. Zobaczymy również, że zamiast wpisywać konkretne wartości w~pola, możemy nakazać komputerowi je obliczać. Ponadto blok może mieć więcej niż jeden parametr. Na przykład blok \code{leć}, znajdujący się mniej więcej w~połowie palety ,,Ruch'', przyjmuje trzy parametry.

Większość bloków komend ma kształt klocków, lecz niektóre, jak \code{powtórz} z~tego samego przykładu, wyglądają jak \emph{klamry}. Większość bloków klamrowych można znaleźć na palecie ,,Kontrola'. Wnętrze klamry jest szczególnym rodzajem pola parametru, który przyjmuje \emph{skrypt} jako parametr. W~przykładowym skrypcie blok \code{powtórz} ma dwa parametry: liczbę $4$ oraz skrypt\nopagebreak

\bigpic{typowy-skrypt-wnetrze}

\section{Duszki i~współbieżność}

Tuż pod sceną znajduje się przycisk ,,nowy duszek''~\inlinepic{../common/btn-new-sprite}. Kliknięcie go spowoduje dodanie nowego duszka do sceny. Pojawi się on w~losowym miejscu na scenie, skierowany w~losową stronę i~zabarwiony na losowy kolor.

Każdy duszek ma swoje własne skrypty. Aby wyświetlić w~obszarze skryptów te należące do konkretnego duszka, należy kliknąć na jego obrazku w~\emph{zagrodzie duszków}, znajdującej się w~prawym dolnym rogu okna. Spróbuj umieścić następujące skrypty w~obszarze skryptów --- po jednym dla każdego duszka:\nopagebreak

\begin{figure}[H]
\begin{minipage}{0.5\textwidth}
\centering
\includegraphics[scale=\defaultGraphicsScale]{duszki-i-wspolbieznosc-1}
\end{minipage}%
\begin{minipage}{0.5\textwidth}
\centering
\includegraphics[scale=\defaultGraphicsScale]{duszki-i-wspolbieznosc-2}
\end{minipage}
\end{figure}

Kiedy klikniemy zieloną flagę~\inlinepic{../common/btn-start}, powinniśmy zobaczyć jak jeden duszek się obraca, podczas gdy drugi porusza się w~tę i~z~powrotem. Ten eksperyment pokazuje, jak różne skrypty mogą być wykonywane jednocześnie (\emph{współbieżnie}). Obracanie się dookoła i~ruch po linii prostej zachodzą jednocześnie. Współbieżność zachodzi również w~przypadku wielu skryptów należących do tego samego duszka. Spróbujmy tego przykładu:\nopagebreak

\begin{figure}[H]
\begin{minipage}{0.5\textwidth}
\centering
\includegraphics[scale=\defaultGraphicsScale]{duszki-i-wspolbieznosc-3}
\end{minipage}%
\begin{minipage}{0.5\textwidth}
\centering
\includegraphics[scale=\defaultGraphicsScale]{duszki-i-wspolbieznosc-4}
\end{minipage}
\end{figure}

Po naciśnięciu spacji duszek powinien zacząć bez końca chodzić w kółko, ponieważ bloki \code{przesuń} i \code{obróć} są wykonywane współbieżnie. (Aby przerwać program, kliknij czerwony czerwony znak ,,stop''~\inlinepic{../common/btn-stop} na prawym brzegu paska narzędzi).

\subsection{Kostiumy i~dźwięki}

Aby zmienić wygląd duszka, należy zaimportować dla niego nowy \emph{kostium}. Są na to trzy sposoby. Najpierw trzeba wybrać duszka z~zagrody. Następnie, w~pierwszej metodzie, klikamy na ikonie pliku~\inlinepic{../common/btn-file} na pasku narzędzi, a~następnie wybieramy polecenie ,,\code{Kostiumy\ldots}''. Ukaże się lista kostiumów z~publicznej biblioteki multimediów, spośród których możemy dokonać wyboru. Drugą metodą jest wybór pliku ze swojego własnego komputera. Należy w~tym celu kliknąć ikonę pliku, a~następnie polecenie ,,\code{Importuj\ldots}''. Można wtedy wybrać plik obrazu w~dowolnym formacie (PNG, JPEG itd.) obsługiwanym przez przeglądarkę. Trzeci sposób jest szybszy jeśli plik, którego chcemy użyć, jest widoczny na pulpicie: po prostu przeciągnij go do okna \Snap{a}. W~każdym z~tych przypadków obszar skryptów zacznie wyglądać mniej więcej tak:\nopagebreak

\bigpic{obszar-skryptow-z-dodatkowym-kostiumem}

Tuż nad tą częścią okna znajdują się trzy zakładki: ,,Skrypty'', ,,Kostiumy'' i~,,Dźwię\-ki''. W~tym momencie aktywna jest karta ,,Kostiumy''. Widzimy na niej \emph{garderobę} duszka i~możemy z~jej poziomu wybrać dla niego kostium --- domyślny kostium żółwia\footnote{Z powodów historycznych, słowem ,,żółw'' nazywamy ruchomy obiekt, który porusza się wykonując program i~rysuje, zostawiając za sobą ślad} lub wybrany wcześniej kostium Alonza. (Alonzo, maskotka \Snap{a}, został nazwany na cześć Alonza Churcha, matematyka, który jako pierwszy wpadł na pomysł, aby procedury traktować na równi z~danymi, co jest najistotniejszą różnicą między \Snap{em} a~Scratchem). Możemy przypisać duszkowi tyle kostiumów ile chcemy, a~potem wybierać, który z~nich założy, albo poprzez kliknięcie w~obrębie garderoby, albo używając w~skrypcie bloku \inlinepic{zmien-kostium-na-zolwia} lub \inlinepic{nastepny-kostium}. (Każdy kostium ma zarówno numer jak i~nazwę. Blok \code{następny kostium} wybiera następny w~kolejności kostium; po ostatnim wybiera z~powrotem kostium numer~1. Żółw, czyli kostium numer~0, jest przez blok \code{następny kostium} ignorowany). Kostium ,,Żółw'' jest jedynym, który zmienia kolor zgodnie z~kolorem pisaka.

Oprócz kostiumów, duszki mogą posiadać \emph{dźwięki}; dźwiękowy odpowiednik garderoby duszka nazywamy jego \emph{szafą grającą}. Można importować pliki dźwiękowe w~dowolnym formacie obsługiwanym przez przeglądarkę. Do odtwarzania dźwięków służą dwa rodzaje bloków. Jeśli skrypt ma się dalej wykonywać podczas odtwarzania, używamy bloku \inlinepic{zagraj-dzwiek-ratunku}. Za to aby poczekać, aż dźwięk się zakończy, zanim skrypt będzie kontynuowany, należy wykorzystać blok \inlinepic{zagraj-dzwiek-ratunku-i-czekaj}.

\subsection{Nadawanie i~odbieranie komunikatów}

Widzieliśmy wcześniej przykład dwóch duszków poruszających się jednocześnie. Jednak w~bardziej interesującym programie duszki na scenie będą wchodzić w~interakcje, abyśmy mogli opowiedzieć przy ich pomocy jakąś historię, zagrać w~grę itd. Czasami jeden duszek będzie musiał nakazać innemu wykonanie jakiegoś skryptu. Oto prosty przykład:\nopagebreak

\begin{figure}[H]
\begin{minipage}{0.5\textwidth}
\centering
\includegraphics[scale=0.4]{../common/boy1-walking}
\end{minipage}%
\begin{minipage}{0.5\textwidth}
\centering
\reflectbox{\includegraphics[scale=0.3]{../common/dog2-c}}
\end{minipage}
\vskip 3ex
\begin{minipage}[t]{0.5\textwidth}
\centering
\vspace{0pt} % REALLY align to top
\includegraphics[scale=\defaultGraphicsScale]{nadawanie-i-odbieranie-komunikatow-1}
\end{minipage}%
\begin{minipage}[t]{0.5\textwidth}
\centering
\vspace{0pt} % REALLY align to top
\includegraphics[scale=\defaultGraphicsScale]{nadawanie-i-odbieranie-komunikatow-2}
\end{minipage}
\end{figure}

Słowo ,,szczekaj'' występujące w~bloku \inlinepic{nadaj-szczekaj-do-wszystkich-i-czekaj} to pierwszy lepszy wyraz, który przyszedł mi do głowy. Jedną z~opcji, które ukazują się po kliknięciu strzałki w~dół obok tego pola parametru (i~jedyną początkowo dostępną), jest ,,\code{nowy}''. Po jej wybraniu \Snap{} pyta o~nazwę komunikatu. Kiedy wspomniany blok zostanie wykonany, wybrany komunikat zostaje wysłany do \emph{każdego} duszka --- stąd też określenie ,,nadaj do wszystkich''. Jednak w~naszym przykładzie tylko jeden duszek ma skrypt, który jest wywoływany w~momencie nadania tego komunikatu --- jest nim pies. Ponieważ skrypt chłopca wykorzystuje blok \code{nadaj do wszystkich i~czekaj} zamiast \code{nadaj do wszystkich}, chłopiec nie przechodzi do następującego po nim bloku \code{powiedz}, dopóki skrypt psa się nie skończy. Z~tej przyczyny dwa duszki mówią na zmianę, a~nie jednocześnie.

Warto przy okazji zwrócić uwagę na to, że pierwsze pole parametru na bloku \code{powiedz} nie jest owalne, lecz prostokątne. Oznacza to, że parametr ten może być dowolnym łańcuchem znaków (tekstem), nie tylko liczbą. W~polach parametrów typu tekstowego spacje ukazują się jako brązowe kropki, abyśmy mogli policzyć liczbę odstępów między wyrazami. Co ważniejsze, możemy dzięki temu odróżnić pusty łańcuch od złożonego z~samych spacji. Brązowe kropki \emph{nie będą} widoczne na scenie, kiedy blok zostanie wykonany.

Scena ma swój własny obszar skryptów. Możemy wyświetlić jej szczegóły klikając ikonę ,,Scena'' po lewej stronie zagrody duszków. W~przeciwieństwie do duszków scena się nie porusza. Zamiast kostiumów ma \emph{tła} --- obrazki wypełniające cały obszar sceny. Duszki rysowane są na aktualnym tle. W~skomplikowanych projektach często wygodnie jest użyć skryptu sceny do koordynacji działań poszczególnych części programu.

\section{Zagnieżdżanie duszków: kotwice i~części}
\label{sec:zagnieżdżanie-duszków}

Czasem dobrze jest stworzyć swego rodzaju ,,nadduszka'', złożonego z~kawałków, które poruszają się razem, ale mogą być osobno względem siebie ustawiane. Klasycznym przykładem może być ciało człowieka złożone z~tułowia, kończyn i~głowy. \Snap{} pozwala nam uczynić jednego z~duszków \emph{kotwicą} złożonego obiektu, a~resztę --- jego \emph{częściami}. Aby zagnieździć w~ten sposób duszki, należy przeciągnąć z~zagrody ikonę duszka, który ma zostać \emph{częścią} złożonego obiektu na znajdującego się na scenie (nie w~zagrodzie!) duszka, który zostanie \emph{kotwicą}.

Zagnieżdżone duszki --- zarówno kotwice jak i części --- mają w zagrodzie specjalne oznaczenia:\nopagebreak

\bigpic{zagniezdzone-duszki-w-zagrodzie}

W~tym przypadku chcielibyśmy animować rękę Alonza. (Ręka została pokolorowana na zielono, aby uwypuklić zależność między dwoma duszkami, choć w~prawdziwym projekcie miałyby one raczej ten sam kolor). ,,Duszek1'', reprezentujący ciało Alonza, jest kotwicą; ,,Duszek2'' to ręka. Ikona duszka-kotwicy zawiera w~dolnej części do trzech miniatur doczepionych do niego duszków-części. Z~kolei na ikonie każdej z~części widać pomniejszony obrazek duszka-kotwicy w~lewym górnym rogu, w~prawym górnym zaś --- \emph{przełącznik synchronizacji obrotów}. Początkowo, jak widać na rysunku powyżej, jest on tak ustawiony, aby obrót kotwicy powodował zarówno orbitowanie części wokół niej, jak i~obrót części dookoła swojej własnej osi. Po kliknięciu przełącznik zmienia kształt z~okrągłej strzałki na prostą, co oznacza, że od tej pory obrót duszka-kotwicy będzie powodował jedynie zmianę pozycji przymocowanych do niego części, ale nie będą się one obracać wokół własnej osi. (Części mogą również obracać się niezależnie, przy pomocy bloków \code{obróć}). Każda zmiana pozycji lub rozmiaru kotwicy jest propagowana na wszystkie części.

\begin{figure}[H]
\centering
\includegraphics[scale=\defaultGraphicsScale]{komenda-machania-reka}%
\hspace{2em}%
\includegraphics[scale=0.4]{../common/alonzo-waving}
\end{figure}

\section{Bloki funkcji i~wyrażenia}

Jak dotąd używaliśmy dwóch rodzajów bloków: ,,czapek'' i~komend. Kolejnym rodzajem jest blok \emph{funkcji}, który ma owalny kształt: \inlinereporterpic{pozycja-x}. Nazywamy go ,,blokiem funkcji'', ponieważ --- podobnie jak funkcja w~matematyce --- kiedy zostaje wykonany, zamiast przeprowadzać jakąś czynność, zwraca wartość, która może zostać użyta jako parametr w~innym bloku. Jeśli przeciągniemy sam blok funkcji do obszaru skryptów i~klikniemy go, obok pokaże się dymek z~wartością zwróconą przez tę funkcję:\nopagebreak

\bigpic{pozycja-x-zwraca-liczbe}

Kiedy przeciągamy blok funkcji nad polem parametru należącym do innego bloku, wokół tego pola pojawia się biała otoczka, analogicznie do sytuacji, w~której łączymy bloki komend i~pojawia się biała linia. Oto przykładowy skrypt wykorzystujący funkcję:\nopagebreak

\begin{figure}[H]
\centering
\includegraphics[scale=\defaultGraphicsScale]{przykladowy-skrypt-wykorzystujacy-funkcje}%
\hspace{2em}%
\includegraphics{../common/turtle-says-its-position}
\end{figure}

Funkcja \code{pozycja X} nadaje tu wartość pierwszemu parametrowi bloku \code{powiedz}. Pozycja X~duszka to inaczej jego współrzędna pozioma. Określa ona, jak daleko w~lewo (jeśli jest liczbą ujemną) lub w~prawo (jeśli dodatnią) znajduje się duszek w~stosunku do środka sceny. Analogicznie, pozycja Y~to współrzędna pionowa, mierzona ilością kroków w~górę (wartości dodatnie) lub w~dół od środka (wartości ujemne).

Przy pomocy funkcji z palety ,,Wyrażenia'' możemy wykonywać obliczenia:\nopagebreak

\begin{figure}[H]
\centering
\includegraphics[scale=\defaultGraphicsScale]{uzycie-funkcji-do-obliczen}%
\hspace{2em}%
\includegraphics{../common/turtle-says-its-rounded-position}
\end{figure}

Blok \code{zaokrąglij} zaokrągla $35.3905\ldots$ do $35$, a~blok~\code{+} dodaje do tej liczby $100$. Nawiasem mówiąc, choć blok \code{zaokrąglij} znajduje się na palecie ,,Wyrażenia'', podobnie jak~\code{+}, to w~tym skrypcie ma on jaśniejszy kolor i~czarne litery. To dlatego, że \Snap{} używa na przemian ciemnych i~jasnych odcieni kolorów, kiedy zagnieżdżamy w~sobie bloki z~tej samej palety:\nopagebreak

\bigpic{kolorowanie-w-zebre}

Takie \emph{kolorowanie w~zebrę} poprawia czytelność programu. Blok funkcji wraz z~parametrami, a~być może również innymi blokami funkcji, na przykład \inlinepic{zaokraglij-pozycja-x-plus-100}, nazywamy \emph{wyrażeniem}.

\section{Predykaty i~obliczenia warunkowe}

Większość funkcji zwraca albo liczbę, jak \inlinereporterpic{plus}, lub łańcuch tekstowy, jak \inlinereporterpic{polacz-witaj-swiecie}. \emph{Predykat} to specjalny rodzaj funkcji, która zawsze zwraca jedną z dwojga wartości: \code{prawdę} lub \code{fałsz}. Predykaty mają kształt sześciokątów:\nopagebreak

\bigpic{przycisk-myszy-nacisniety}

\begin{sloppypar}
Specjalny kształt jest oznaką, że predykaty nie mają z~reguły sensu w~tych polach parametrów, które oczekują liczby lub tekstu. Raczej nie napisalibyśmy \inlinepic{przesun-o-przycisk-myszy-nacisniety-krokow}, choć gdybyśmy się uparli, \Snap{} by nam na to pozwolił, co widać na załączonym obrazku. W~typowych sytuacjach predykaty umieszczamy w~specjalnych sześciokątnych polach parametrów takich jak to:\nopagebreak
\end{sloppypar}

\bigpic{jezeli-to}

Klamra \code{jeżeli --- to} wykonuje obejmowany przez nią fragment skryptu wtedy i~tylko wtedy, gdy wyrażenie w~jej sześciokątnym polu parametru jest prawdziwe, czyli zwraca wartość \code{prawda}.\nopagebreak

\bigpic{predykaty-i-obliczenia-warunkowe-1}

Poniższy blok jest bardzo użyteczny w~animacjach. Wykonuje on skrypt będący jego parametrem \emph{wielokrotnie}, dopóki predykat nie zostanie spełniony:\nopagebreak

\bigpic{predykaty-i-obliczenia-warunkowe-2}

Jeśli pracując nad projektem, będziemy chcieli tymczasowo pominąć niektóre komendy w~skrypcie, lecz nie będziemy chcieli zapomnieć, gdzie było ich miejsce, możemy użyć następującej sztuczki:\nopagebreak

\bigpic{predykaty-i-obliczenia-warunkowe-3}

Czasami potrzeba wykonać tę samą czynność bez względu na to, czy jakiś warunek zachodzi czy nie, za to z~różnymi parametrami dla obu tych przypadków. Można do tego użyć bloku \emph{funkcji} \code{if}:\footnote{\onehalfspacing Jeśli nie widzisz go w~palecie ,,Kontrola'', kliknij przycisk ,,Plik''~\inlinepic{../common/btn-file} na pasku narzędzi i~wybierz polecenie ,,Importuj narzędzia''.}\footnote{Niestety, podobnie jak pozostałe dodatkowe narzędzia i~biblioteki bloków, funkcja \code{if --- then --- else} posiada wyłącznie angielską nazwę, bez względu na nasze ustawienia języka. Oznacza ona ,,jeżeli --- to --- w~przeciwnym razie'' --- przyp. tłum.}\nopagebreak

\bigpic{predykaty-i-obliczenia-warunkowe-4}

Wartości \code{prawda} i~\code{fałsz} określa się technicznymi terminami ,,wartość logiczna'' lub ,,wartość boolowska'' (ang. \textit{Boolean}). Ta ostatnia nazwa pochodzi on nazwiska George'a Boole'a, który stworzył opisującą je teorię matematyczną\footnote{Jest to \emph{algebra Boole'a} --- przyp. tłum.}. Uważaj na nazewnictwo --- sześciokątny blok to \emph{predykat}, ale wartość przezeń zwracana to \emph{wartość logiczna}.

Jest jeszcze jedna warta wytłumaczenia niejasność terminologiczna:  W wielu językach programowania nazwa ,,procedura'' jest zarezerwowana dla \emph{komend}, które wykonują jakąś czynność, zaś nazwa ,,funkcja'' --- dla części programów zwracających wartość (\emph{funkcji} i \emph{predykatów}). W tym podręczniku \emph{procedury} to dowolne składniki programu, zarówno te zwracające, jak i nie zwracające wartości. Podobnie jak komendy, również funkcje i predykaty będziemy nazywać procedurami. Słowa ,,typ proceduralny'' będą skrótem myślowym dla słów ,,typ komendowy, funkcyjny lub predykatowy''.

\section{Zmienne}

Wypróbujmy następujący skrypt:\footnote{Blok \code{for}\footnotemark{} również znajduje się w~bibliotece narzędzi; użyj polecenia ,,\code{Importuj narzędzia}'' z~menu ,,Plik'', jeśli nie widzisz ich na palecie ,,Kontrola''.}\footnotetext{Słowa \code{for i = 1 to 10} oznaczają ,,dla $i=1$ do $10$'' --- przyp. tłum.}\nopagebreak

\bigpic{skrypt-kwadratowej-spirali}

Parametr bloku \code{przesuń} ma postać pomarańczowego owalu. Aby go tam umieścić, należy przeciągnąć taki sam owal będący częścią bloku \code{for}:\nopagebreak

\bigpic{przeciaganie-zmiennej}

Ten owal to \emph{zmienna} --- symboliczna nazwa reprezentująca jakąś wartość. Powyższy rysunek przedstawia sytuację sprzed zmiany drugiego parametru liczbowego bloku \code{for} z~domyślnego $10$ na $200$ oraz przeciągnięcia do jego środka bloku \code{obróć}. Blok \code{for} wykonuje swój parametr skryptowy wielokrotnie, podobnie jak \code{powtarzaj}, lecz przed każdym razem zapisuje liczbę do zmiennej~\code{i}, zaczynając od swojego pierwszego parametru liczbowego, dodając~$1$ przy każdym powtórzeniu, aż dojdzie do liczby z~drugiego parametru liczbowego. W~tym przypadku będziemy mieć $200$ powtórzeń, najpierw dla $\code{i}=1$, potem dla $\code{i}=2$, następnie $3$ i~tak dalej, aż do $\code{i}=200$ w~ostatnim powtórzeniu. W~rezultacie każdy blok \code{przesuń} rysuje coraz to dłuższy segment łamanej, co nadaje jej wygląd zbliżony do spirali. (Możesz spróbować ze skrętem $90$~stopni zamiast $92$; zobaczysz wtedy, dlaczego nazywamy tego rodzaju obraz ,,kwadratową spiralą'').

Zmienna~\code{i}~została utworzona przez blok \code{for} i~może zostać użyta wyłącznie wewnątrz jego klamry. Nawiasem mówiąc, jeśli nie spodoba nam się nazwa~\code{i}, możemy ją zmienić klikając pomarańczowy owal bez przeciągania go. Pokaże się wtedy okno dialogowe, do którego można wpisać inną nazwę:\nopagebreak

\bigpic{dialog-nazwy-zmiennej-skryptu}

Nazwa~,,\code{i}'' nie mówi nam zbyt wiele; można by tu użyć słowa ,,długość'', aby podkreślić znaczenie zmiennej. Nazwa~,,\code{i}'' jest popularna, gdyż w~matematyce istnieje tradycja używania liter od~\code{i} do~\code{n} dla liczb całkowitych. W~językach programowania nie musimy się jednak ograniczać do jednoliterowych nazw zmiennych.

\subsection{Zmienne globalne}

Możemy ,,ręcznie'' tworzyć zmienne, których widoczność nie jest ograniczona do jednego bloku. Aby to zrobić, należy kliknąć przycisk ,,\code{Stwórz zmienną}'' w~górnej części palety ,,Dane'':\nopagebreak

\bigpic{stworz-zmienna}

Otworzy się okno dialogowe, które pozwala nadać nazwę nowej zmiennej:\nopagebreak

\bigpic{dialog-nazwy-zmiennej}

Okno to pozwala również wybrać, czy zmienna ma być dostępna dla wszystkich duszków (co jest pożądane w~większości przypadków), czy ma być ona widoczna tylko dla aktualnego duszka. Postępujemy tak, jeśli mamy zamiar dać wielu duszkom własne zmienne o~\emph{tej samej nazwie}. Możemy potem kopiować skrypty między duszkami, przeciągając je z~obszaru skryptów aktualnego duszka na obrazek innego duszka w~zagrodzie duszków. Dzięki temu różne duszki będą wykonywać nieco inne rzeczy w~momencie wykonywania tego skryptu, ponieważ każdy z~nich będzie miał w~tej zmiennej zapisaną inną wartość.

Jeśli nadamy zmiennej nazwę ,,imię'', paleta ,,Dane'' będzie wyglądać następująco:\nopagebreak

\bigpic{zmienna-na-palecie}

Widzimy teraz przycisk ,,\code{usuń zmienną}'', a~także pomarańczowy owal z~nazwą zmiennej, taki sam jak owal na bloku \code{for}. Zmienną możemy przeciągnąć do dowolnego skryptu w~obszarze skryptów. Obok owalu widzimy pole wyboru; jest ono początkowo zaznaczone, dzięki czemu na scenie widoczny jest \emph{podgląd zmiennej}:\nopagebreak

\bigpic{podglad-zmiennej}

Kiedy nadamy zmiennej jakąś wartość, pojawia ona się w~polu podglądu.

Ale \emph{jak} nadać zmiennej wartość? Należy użyć do tego bloku \code{ustaw}:\nopagebreak

\bigpic{zapytaj-i-ustaw}

Zauważ, że \emph{nie przeciągamy} owalu zmiennej do bloku \code{ustaw}! Aby wybrać z~listy dostępnych zmiennych, należy kliknąć strzałkę w~dół przy pierwszym polu parametru tegoż bloku.

\subsection{Zmienne skryptu}

W~poprzednim przykładzie przeprowadzaliśmy interakcję z~użytkownikiem i~chcieliśmy zapamiętać jego imię na potrzeby całego projektu. To dobry przykład sytuacji, w~której właściwe jest użycie zmiennej \emph{globalnej} (z~rodzaju tych, które tworzymy przyciskiem ,,\code{Stwórz zmienną}''). Innym typowym przykładem jest zmienna ,,\code{wynik}'' w~projekcie gry. Czasem jednak potrzebujemy zmiennej tylko tymczasowo podczas wykonywania któregoś ze skryptów. W~takim przypadku możemy użyć bloku \code{zmienne skryptu} aby ją utworzyć:\nopagebreak

\begin{figure}[H]
\begin{minipage}{0.5\textwidth}
\centering
\includegraphics[scale=\defaultGraphicsScale]{skrypt-skrecajacej-linii}%
\end{minipage}%
\begin{minipage}{0.5\textwidth}
\centering
\includegraphics{../common/wiggling-line}
\end{minipage}%
\end{figure}

Podobnie jak w~przypadku bloku \code{for}, aby zmienić nazwę zmiennej, wystarczy kliknąć pomarańczowy owal w~bloku \code{zmienne skryptu} bez przeciągania. Możemy również stworzyć więcej tymczasowych zmiennych klikając strzałkę w~prawo na końcu bloku. Spowoduje to dodanie kolejnego owalu zmiennej:\nopagebreak

\bigpic{zmienne-skryptu-a-b-c}

\section{I tak dalej}

Niniejszy podręcznik nie opisuje szczegółowo każdego bloku. Jest wiele bloków związanych z~ruchem, dźwiękiem, kostiumami, efektami graficznymi i~tak dalej. Ich przeznaczenie można poznać eksperymentalnie, ale także poprzez lekturę ,,ekranów pomocy''\footnote{Niestety, są one dostępne wyłącznie w~języku angielskim --- przyp. tłum.}, które można obejrzeć klikając interesujący nas blok albo prawym przyciskiem myszy, albo lewym z~wciśniętym jednocześnie klawiszem Ctrl, a~następnie wybierając z~menu polecenie ,,\code{pomoc\ldots}''. Jeśli zapomnisz, w~której palecie znajduje się potrzebny Ci blok, ale pamiętasz choćby część nazwy, wciśnij Ctrl-F i~wpisz ją w~polu tekstowym, które pojawi się w~obszarze palet.

\chapter{Zapisywanie i otwieranie projektów i multimediów}
\label{ch:zapisywanie-i-otwieranie-projektów-i-multimediów}

Kiedy już stworzymy jakiś projekt, dobrze by było móc go zapisać, aby mieć go pod ręką, kiedy ponownie uruchomimy \Snap{a}. Jest na to kilka sposobów. Możemy zapisać projekt na swoim komputerze albo na stronie internetowej \Snap{a}. Zaletą zapisywania w~sieci jest dostęp do projektów nawet podczas używania innego komputera lub urządzenia mobilnego, takiego jak tablet czy smartfon. Z~kolei zapisywanie na własnym komputerze pozwala na dostęp do zapisanych projektów w~przypadku braku dostępu do sieci, na przykład w~czasie podróży pociągiem lub samolotem. Dlatego właśnie mamy wiele sposobów na zapisanie projektu.

\section{Zapisywanie na komputerze}

Istnieją dwa różne sposoby na zapisanie projektu lub pliku multimedialnego (na przykład kostiumu) na własnym komputerze. Ta złożoność wynika z~tego, że JavaScript, w~którym \Snap{} jest zaimplementowany, celowo ogranicza wpływ programów wykonywanych przez przeglądarkę na komputer użytkownika. Jest to pożyteczne, ponieważ dzięki temu możemy z~pełnym zaufaniem uruchamiać w~\Snap{ie} cudze projekty --- bez obawy, że usuną nam wszystkie pliki lub zainfekują komputer wirusem. Jednak mechanizm ten nieco utrudnia pracę.

\subsection{Localstore}

{\Huge \TODO{}}

\subsection{Eksport do pliku XML}
\label{subsec:eksport-do-pliku-xml}

Drugi sposób na zapisanie projektu na komputerze ma dwa etapy, ale nie ma ograniczeń związanych z~użyciem localstore (\TODO{zmień nazwę}). Projekty zapisane w~ten sposób są zwykłymi plikami na dysku komputera i~można je wysłać do znajomych oraz otworzyć w~dowolnej przeglądarce. Ponadto ich rozmiar nie jest ograniczony.

Z~menu ,,Plik''~\inlinepic{../common/btn-file} wybieramy ,,Eksportuj projekt\ldots''. Okno \Snap{a} zniknie i~zostanie zastąpione przez ekran wypełniony ,,krzakami''. Bez paniki! Tak ma być. To, co widzimy, to zapis projektu w~notacji zwanej XML. Głównym powodem, dla którego wygląda on jak stek bzdur, jest to, że zawiera on zakodowane obrazki i~inne multimedia zawarte w~projekcie. Jeśli się dobrze wpatrzymy, same skrypty są jako tako czytelne, choć nie wyglądają jak bloki \Snap{a}. Przeglądarka otworzyła dla tekstu XML nową kartę; okno \Snap{a} jest wciąż otwarte, ukryte pod spodem.

Jednak tekst XML nie jest po to, abyśmy go czytali. Kiedy już mamy otwartą tę kartę, możemy użyć polecenia ,,Zapisz'' w~przeglądarce. Można je uruchomić z~poziomu menu przeglądarki lub przy pomocy skrótu klawiszowego --- zazwyczaj Command-S (Mac) lub Ctrl-S (wszędzie indziej). Wybieramy nazwę dla pliku, a~przeglądarka zapisze go w~folderze ,,Pobrane''\footnote{lub w~innym miejscu, które wskażemy w~oknie zapisu --- przyp. tłum.}. Na koniec zamykamy kartę z~plikiem XML i~wracamy do środowiska \Snap{}.

\section{Zapisywanie w~chmurze}

Inną możliwością jest zapisanie projektu ,,w~chmurze'', na stronie internetowej \Snap{a}. Aby to zrobić, musimy najpierw założyć tam konto. Klikamy na przycisku ,,Chmura'' (\,\inlinepic{../common/btn-cloud}\,) na pasku narzędzi, a~następnie wybieramy polecenie ,,Rejestracja\ldots''. Pokaże się następujące okno:\nopagebreak

\bigpic{rejestracja}

Należy teraz wybrać nazwę użytkownika, którą będziemy się legitymować w~obrębie strony, taką jak na przykład \code{Jens} lub \code{bh}. Jeśli jesteś użytkownikiem Scratcha, możesz również użyć na stronie \Snap{a} swojej nazwy użytkownika z~serwisu Scratch. Jeśli jesteś dzieckiem, nie wybieraj nazwy użytkownika, która zawierałaby twoje nazwisko; imiona i~inicjały są akceptowalne. Nie wybieraj też nazwy, której wstydziłbyś się pokazać innym użytkownikom lub rodzicom! Jeśli wybrana przez Ciebie nazwa jest już zajęta, będziesz musiał wybrać inną.

Będziesz musiał też podać swój miesiąc i~rok urodzenia. \Snap{} używa tych informacji tylko i~wyłącznie aby zdecydować, czy zapytać Cię o~Twój własny adres e-mail, czy też o~adres rodzica. Jeśli jesteś dzieckiem, nie powinieneś zakładać konta w~żadnym serwisie internetowym --- wliczając w~to \Snap{a} --- bez zgody rodzica. \Snap{} nie będzie przechowywać Twojej daty urodzenia na serwerze; zostanie ona użyta tylko na Twoim własnym komputerze podczas procedury rejestracji. Program nie wymaga podania \emph{dokładnej} daty, nawet ten jeden raz, ponieważ jest to ważna dana osobowa.

Po kliknięciu przycisku OK, na podany adres e-mail zostanie wysłana wiadomość z~początkowym hasłem do nowego konta. \Snap{} będzie przechowywać Twój adres, aby w~przypadku zapomnienia hasła móc wysłać Ci link pozwalający je zresetować. Ponadto \Snap{} wyśle Ci wiadomość, jeśli Twoje konto zostanie zawieszone za naruszenie regulaminu. Twój adres nie będzie wykorzystywany do innych celów. Nie będziesz dostawać żadnego rodzaju wiadomości marketingowych poprzez tę stronę --- ani od zespołu \Snap{a}, ani od osób trzecich. Jeśli mimo to wahasz się przed podaniem tej informacji, poszukaj w~internecie pod hasłem ,,tymczasowy e-mail''.

Na koniec powinieneś przeczytać regulamin usługi i~wyrazić zgodę na jego postanowienia. Oto ich krótkie streszczenie: nie przeszkadzaj korzystać ze \Snap{a} innym użytkownikom, nie umieszczaj cudzych utworów chronionych prawem autorskim ani żadnych danych osobowych w~projektach udostępnionych innym. Zespół \Snap{a} nie bierze również odpowiedzialności za szkody, jeśli coś pójdzie nie tak. (To nie znaczy, że można by \emph{oczekiwać}, że coś złego się stanie --- ponieważ \Snap{} używa JavaScriptu wewnątrz przeglądarki, jest mocno izolowany od całej reszty komputera. Jednak prawo jest prawem i~musimy to wyraźnie zaznaczyć.)

Po utworzeniu konta, możemy się na nie zalogować używając polecenia ,,Logowanie\ldots'' z~menu ,,Chmura'':\nopagebreak

\bigpic{zaloguj-sie}

Użyj ustalonej wcześniej nazwy użytkownika i~hasła. Jeśli zaznaczysz pole ,,Zapamiętaj mnie na tym komputerze'', zostaniesz zalogowany automatycznie, kiedy następnym razem uruchomisz \Snap{a} w~tej samej przeglądarce na tym samym komputerze. Zaznacz to pole jeśli używasz swojego własnego komputera i~nie udostępniasz go innym. Nie zaznaczaj go, jeśli używasz publicznie dostępnego komputera w~bibliotece, szkole itp.

Kiedy już się zalogujemy, możemy wybrać opcję ,,Chmura'' w~oknie ,,Zapisz projekt'' ze strony \TODO{Link do obrazka}. Wpisujemy nazwę projektu, a~także opcjonalnie notatki, tak jak w~przypadku zapisywania w~Localstore \TODO{nazwa}. Jednak tym razem nasz projekt będzie zapisany online i~będzie mógł zostać wczytany z~dowolnego miejsca z~dostępem do sieci.

\section{Wczytywanie zapisanych projektów}

Kiedy już zapisaliśmy projekt, chcielibyśmy go wczytać z~powrotem do \Snap{a}. Są na to dwa sposoby:

\begin{enumerate}
\item Jeśli zapisałeś projekt w~Localstore \TODO{nazwa} lub na swoim koncie w~serwisie \Snap{}, użyj polecenia ,,Otwórz\ldots'' z~menu ,,Plik''. Wybierz opcję ,,Przeglądarka'' lub ,,Chmura'', a~potem wybierz swój projekt z~listy i~kliknij OK. Trzecia opcja --- ,,Przykłady'' --- pozwala wybrać spośród dostarczonych przez zespół \Snap{a} przykładowych projektów. Możemy dowiedzieć się o~czym jest każdy z~nich klikając na nim i~czytając jego notatki.

\item Jeśli zapisałeś projekt w~formacie XML na komputerze, kliknij ,,Importuj\ldots'' w~menu ,,Plik''. Przeglądarka pokaże zwyczajne okno otwierania pliku, przy pomocy którego możesz wskazać projekt, podobnie jak w~innych programach. Możesz również znaleźć swój plik XML z~poziomu pulpitu, a~potem po prostu przeciągnąć go do okna \Snap{a}.
\end{enumerate}

Druga z powyższych technik pozwala również importować do projektu multimedia (kostiumy i dźwięki). Wystarczy wybrać ,,Importuj...'', a potem wskazać obrazek lub dźwięk zamiast pliku XML.

\Snap{} potrafi również importować projekty stworzone przy pomocy BYOB~3.0, Scratcha~1.4 lub Scratcha~2.0 (to ostatnie z~pewnymi utrudnieniami; zob. naszą stronę internetową). Niemal wszystkie takie projekty pracują poprawnie pod \Snap{em}, z~wyłączeniem niewielkiej liczby niekompatybilnych bloków. Projekty z~BYOB~3.1 również działają, pod warunkiem, że nie korzystają z~pierwszoklasowych duszków; w~\Snap{ie}~4.1 będą już działać wszystkie projekty z~BYOB~3.1.

\chapter{Tworzenie własnych bloków}

\Snap{} pierwotnie nazywał się BYOB, co oznacza ,,Zbuduj Swoje Własne Bloki'' (ang. \textit{Build Your Own Blocks}). Była to pierwsza i~do dziś najważniejsza cecha, którą dodaliśmy do Scratcha. (Nazwa została zmieniona, gdyż niektórzy nauczyciele nie podzielali naszego poczucia humoru.\footnote{Skrót BYOB oznacza tak naprawdę \textit{bring your own beer}, czyli ,,przynieś własne piwo'' i~oznacza, że zostajemy zaproszeni na przyjęcie, ale oczekuje się od nas, że przyniesiemy ze sobą alkohol --- przyp. tłum.} Cóż, czasem trzeba wiedzieć kiedy się poddać.) Nowy Scratch~2.0 również częściowo obsługuje tworzenie własnych bloków.

\section{Proste bloki}

Na palecie ,,Dane'', w~pobliżu dolnej krawędzi, znajdziemy przycisk ,,nowy blok''.\nopagebreak

\begin{figure}[H]
\centering
\input{paleta-dane.pdf_tex}
\end{figure}

Po kliknięciu tego przycisku pokaże się okno dialogowe, przy pomocy którego możemy wybrać nazwę i~kształt bloku, a~także jego paletę (a~zarazem kolor). Możemy także zdecydować, czy blok ten będzie dostępny dla wszystkich duszków, czy tylko dla aktualnego duszka i~jego klonów. Uwaga: Możemy również wywołać okno ,,nowy blok'' poprzez kliknięcie na tle obszaru skryptu prawym przyciskiem (lub lewym z~wciśniętym klawiszem Ctrl), a~następnie wybranie z~menu polecenia ,,buduj nowy blok\ldots''.

\bigpic{nowy-blok}

W~tym oknie dialogowym możemy wybrać paletę, kształt i~nazwę bloku. Poza jednym wyjątkiem, każda paleta ma przyporządkowany jeden kolor, np. wszystkie bloki z~palety ,,Ruch'' są niebieskie. Jednak paleta ,,Dane'' zawiera zarówno pomarańczowe bloki związane ze zmiennymi, jak i~czerwone, związane z~listami. Oba kolory są dostępne; jest również opcja ,,Inne'', przy pomocy której w~palecie ,,Dane'' tworzymy szare bloki, które nie pasują do żadnej z~powyższych kategorii.

Istnieją trzy kształty bloków, zgodnie z~konwencją która powinna być znana użytkownikom Scratcha: bloki w~kształcie puzzli to komendy --- nie zwracają one wyników. Bloki owalne to funkcje, które zwracają wyniki, a~sześciokątne to predykaty, czyli, innymi słowy, funkcje zwracające wyniki typu logicznego (prawdę lub fałsz).

Załóżmy, że chcemy utworzyć blok o~nazwie ,,kwadrat'', który rysuje kwadrat. W~oknie ,,nowy blok'' wybieramy opcje ,,Ruch'' i~,,Komenda'', a~następnie wpisujemy ,,\code{kwadrat}'' w~pole nazwy. Po kliknięciu przycisku OK przechodzimy do edytora bloków. Korzystamy z~niego w~taki sam sposób, jak z~obszaru skryptów w~głównym oknie. Jedyna różnica polega na tym, że blok-kapelusz na górze skryptu, zamiast nosić nazwę w~rodzaju ,,\code{kiedy zostanę kliknięty}'', zawiera obraz bloku, który budujemy. Ten kapelusz nazywamy \emph{prototypem}\footnote{W~tym znaczeniu słowo ,,prototyp'' nie jest związane z~omawianym później \emph{programowaniem obiektowym opartym na prototypach}.} nowo tworzonego bloku. Aby zaprogramować działanie własnego bloku, układamy inne bloki pod kapeluszem, a~następnie klikamy przycisk OK:\nopagebreak

\bigpic{przeciaganie-bloku-do-edytora-blokow}
\bigpic{edytor-blokow-kwadrat}

Nasz blok pojawi się na dole palety ,,Ruch''. Oto on wraz z~rezultatem jego użycia:\nopagebreak

\begin{figure}[H]
\centering
\includegraphics[scale=\defaultGraphicsScale]%
	{blok-kwadrat-na-palecie}
\includegraphics[scale=\defaultGraphicsScale]{../common/square}
\end{figure}

\subsection{Własne bloki z parametrami}

Załóżmy jednak, że chcielibyśmy móc rysować kwadraty o~różnych rozmiarach. Klikamy na bloku prawym przyciskiem lub lewym z~wciśniętym klawiszem Ctrl. Kiedy wybierzemy polecenie ,,\code{edytuj\ldots}'', otworzy się edytor bloków. Zwróć uwagę na symbole plusów przed i~po słowie \code{kwadrat} w~bloku prototypu. Jeśli zatrzymasz wskaźnik myszy nad jednym z~nich, zostanie on podświetlony:\nopagebreak

\bigpic{prototyp-bloku-z-podswietlonym-plusem}

Klikamy prawy plus. Pojawi się okno dialogowe ,,Utwórz nazwę parametru'':\nopagebreak

\bigpic{utworz-nazwe-parametru}

Wpisujemy nazwę ,,\code{wielkość}'' i~klikamy przycisk OK. Dialog ten ma więcej opcji; możemy wybrać ,,\code{Tekst tytułowy}'', aby dodać słowa do nazwy bloku, tak aby po polu parametru następował tekst, podobnie jak w~bloku ,,\code{przesuń o~(\,)~kroków}''. Możemy też użyć bardziej kompleksowego okna z~wieloma opcjami dotyczącymi naszego pola parametru; zostawmy to jednak na później. Po kliknięciu OK ujrzymy w~prototypie bloku nowy parametr:\nopagebreak

\bigpic{skrypt-bloku-kwadrat-z-parametrem-rozmiar}

Teraz już możemy przeciągnąć pomarańczowy owal zmiennej do skryptu, a~następnie kliknąć przycisk OK w~oknie edytora bloków:\nopagebreak

\bigpic{przeciaganie-parametru-bloku}

Nasz blok wraz z~polem parametru jest teraz widoczny na palecie ,,Ruch'':\nopagebreak

\bigpic{blok-kwadrat}

Możemy narysować kwadrat dowolnego rozmiaru wpisując długość boku w~polu parametru i~uruchamiając blok jak zazwyczaj, poprzez kliknięcie na nim lub umieszczenie go w~skrypcie.

\section{Rekurencja}

Ponieważ nowy blok pojawił się na palecie, kiedy tylko \emph{zaczęliśmy} go tworzyć, możemy programować bloki rekurencyjne --- czyli takie, które wywołują same siebie. W tym celu należy przeciągnąć blok do jego własnej definicji:\nopagebreak

\begin{figure}[H]
\centering
\includegraphics[scale=\defaultGraphicsScale]{skrypt-bloku-drzewo}\\
\begin{minipage}{0.5\textwidth}
\centering
\includegraphics[scale=\defaultGraphicsScale]{blok-drzewo-w-skrypcie}
\end{minipage}%
\begin{minipage}{0.5\textwidth}
\centering
\includegraphics[scale=\defaultGraphicsScale]{../common/tree}
\end{minipage}%
\end{figure}

Jeśli rekurencja jest dla Ciebie czymś nowym, oto kilka wskazówek: Kluczowym składnikiem rekurencji jest \emph{przypadek bazowy}, a~więc jakiś minimalny problem, który nasz blok może rozwiązać od razu, bez wywoływania samego siebie. W~naszym przykładzie jest to przypadek $\code{głębokość}=0$, w~którym blok nie robi zupełnie nic, co gwarantuje instrukcja \code{if}, obejmująca w~całości jego treść. Bez przypadku bazowego blok rekurencyjny wykonywałaby się w~nieskończoność\footnote{Mówiąc ściśle: dopóki nie zabraknie pamięci, bowiem każde zagnieżdżone wywołanie bloku rekurencyjnego zużywa jej odrobinę, zwalniając ją dopiero po zakończeniu wykonywania, co w~tym przypadku nigdy by nie nastąpiło. W~przeciwieństwie do wielu innych języków programowania, \Snap{} jest jednak na tyle powolny, że prawdopodobnie trudno by było doczekać się wyczerpania pamięci bez zauważenia wcześniej, że coś jest nie tak --- przyp. tłum.}, wywołując w~kółko sam siebie.

Staraj się nie myśleć o~tym, jaką dokładnie sekwencję kroków wykonuje komputer przy uruchamianiu programu rekurencyjnego. Zamiast tego wyobraź sobie, że wewnątrz komputera żyje mnóstwo małych ludzików. Jeśli jedna z~nich --- nazwijmy ją Dorota --- rysuje drzewo o~rozmiarze 100 i~głębokości 6, musi zatrudnić Dominika, aby zrobił drzewo o~rozmiarze 70 i~głębokości 5, a~następnie Darka do zrobienia kolejnego drzewa o~rozmiarze 70 i~głębokości 5. Dominik z~kolei zatrudnia Dagmarę oraz Darię i~tak dalej. Każdy z~ludzików ma swoje własne zmienne lokalne --- \code{rozmiar} oraz \code{głębokość} --- i~każda z~nich ma swoją własną wartość, potencjalnie inną od zmiennych innych ludzików.

Możesz również tworzyć funkcje rekurencyjne, takie jak na przykład ten blok obliczający silnię:\nopagebreak

\begin{figure}[H]
\begin{minipage}{0.5\textwidth}
\centering
\includegraphics[scale=\defaultGraphicsScale]{skrypt-bloku-silnia}
\end{minipage}%
\begin{minipage}{0.5\textwidth}
\centering
\includegraphics[scale=\defaultGraphicsScale]{silnia-5-z-rezultatem}
\end{minipage}%
\end{figure}

Skupmy się chwilowo na sposobie użycia bloku \code{zwróć}. Kiedy funkcja używa tego bloku, kończy ona swą pracę i~zwraca zadaną wartość; żadne inne bloki w~jej skrypcie nie będą już wykonane. Dlatego też blok \code{jeżeli --- to --- w~przeciwnym razie} z~powyższego skryptu mógłby być po prostu blokiem \code{jeżeli}; wówczas drugi z~bloków \code{zwróć} znajdowałby się pod nim, a~nie wewnątrz niego. Działanie funkcji pozostałoby takie samo, ponieważ w~momencie gdy pierwszy blok \code{zwróć} zostałby wykonany w~przypadku bazowym, działanie funkcji zostałoby zakończone, a~drugi blok \code{zwróć} zostałby zignorowany. Istnieje również komenda \code{zatrzymaj ten blok}, która także kończy wykonywanie aktualnego bloku, a~której używamy do przerywania działania bloków niczego nie zwracających, czyli komend. W~przeciwieństwie do niej blok \code{zatrzymaj ten skrypt} przerywa nie tylko obecne wywołanie bloku, ale cały wywołujący go skrypt, aż do najwyższego poziomu.

A~oto nieco bardziej zwięzły sposób na zapis funkcji silni:\nopagebreak

\bigpic{skrypt-bloku-zwiezla-silnia}

(Jeśli nie widzisz bloku funkcji \code{if} na palecie ,,Kontrola'', kliknij przycisk ,,Plik''~\inlinepic{../common/btn-file} na pasku narzędzi i~wybierz ,,\code{Importuj narzędzia}''.)

Pragnących dowiedzieć się więcej na temat rekursji odsyłamy do książek Erica Robertsa \textit{Thinking Recursively} (ISBN~978-0471816522) oraz nieco bardziej aktualnej \textit{Thinking Recursively in Java} (ISBN~978-0471701460).

\section{Biblioteki bloków}

Kiedy zapisujemy projekt (zob. rozdział~\ref{ch:zapisywanie-i-otwieranie-projektów-i-multimediów} powyżej), wszystkie nasze własne bloki zostaną zapisane razem z~nim. Czasem jednak chcielibyśmy zapisać kolekcję bloków, które mogłyby być przydatne w~więcej niż jednym projekcie. Przykładem może być biblioteka narzędzi, której używaliśmy w~tym podręczniku. Być może nasze bloki implementują jakąś strukturę danych (np. stos, słownik), a~być może stanowią szkielet do budowy wielopoziomowej gry. Taki zbiór bloków nazywamy \emph{biblioteką}.

Do tworzenia bibliotek bloków używamy mechanizmu eksportu do plików XML opisanego na stronie~\pageref{subsec:eksport-do-pliku-xml}, z~tym, że zamiast ,,Eksportuj projekt\ldots'' wybieramy z~menu Plik polecenie ,,Eksportuj bloki\ldots''. Ukaże się wtedy następujące okno:\nopagebreak

\bigpic{dialog-eksportowania-blokow}

Okno to pokazuje wszystkie nasze własne globalne bloki (tj. te dostępne dla wszystkich duszków). Możemy odznaczyć niektóre pola wyboru, aby wybrać które konkretnie bloki mają zostać włączone w~skład naszej biblioteki. Możemy również kliknąć samo okno prawym przyciskiem myszy lub lewym przyciskiem z~klawiszem Ctrl, aby wywołać menu, które pozwoli nam zmienić zaznaczenie wszystkich bloków na raz. Na koniec wciskamy OK. Pokaże się okno zawierające definicje bloków zapisane w~formacie XML. Możemy je teraz zapisać przy pomocy polecenia ,,Zapisz'' w~przeglądarce.

Aby zaimportować bibliotekę bloków, należy użyć polecenia ,,Importuj\ldots'' z~menu ,,Plik''~\inlinepic{../common/btn-file} lub po prostu przeciągnąć plik XML do okna \Snap{a}.

\chapter{First Class Lists}
\section{The list Block}
\section{Lists of Lists}
\section{Functional and Imperative List Programming}
\section{Higher Order List Operations and Rings}
\chapter{Typed Inputs}
\section{Scratch's Type Notation}
\section{The \Snap{} Input Type Dialog}
\subsection{Procedure Types}
\subsection{Pulldown inputs}
\subsection{Input variants}
\subsection{Prototype Hints}
\subsection{Title Text and Symbols}
\chapter{Procedures as Data}
\section{Call and Run}
\subsection{Call/Run with inputs}
\subsection{Variables in Ring Slots}
\section{Writing Higher Order Procedures}
\subsection{Recursive Calls to Multiple-Input Blocks}
\section{Formal Parameters}
\section{Procedures as Data}
1\section{Special Forms}
\subsection{Special Forms in Scratch}
\chapter{Object Oriented Programming}
\section{Local State with Script Variables}
\section{Messages and Dispatch Procedures}
\section{Inheritance via Delegation}
\section{An Implementation of Prototyping OOP}
\chapter{The Outside World}
\section{The World Wide Web}
\section{Hardware Devices}
\section{Date and Time}
\chapter{Continuations}
\section{Continuation Passing Style}
\section{Call/Run w/Continuation}
\subsection{Nonlocal exit}
\chapter{User Interface Elements}
\section{Tool Bar Features}
\subsection{The \Snap{} Logo Menu}
\subsection{The File Menu}
\subsection{The Cloud Menu}
\subsection{The Settings Menu}
\subsection{Stage Resizing Buttons}
\subsection{Project Control Buttons}
\section{The Palette Area}
\subsection{Context Menus for Palette Blocks}
\subsection{Context Menu for the Palette Background}
\section{The Scripting Area}
\subsection{Sprite Appearance and Behavior Controls}
\subsection{Scripting Area Tabs}
\subsection{Scripts and Blocks Within Scripts}
\subsection{Scripting Area Background Context Menu}
\subsection{Controls in the Costumes Tab}
\subsection{The Paint Editor}
\subsection{Controls in the Sounds Tab}
\section{Controls on the Stage}
\section{The Sprite Corral and Sprite Creation Buttons}

\end{document}
